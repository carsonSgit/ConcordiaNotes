\documentclass{article}
\usepackage{amsmath}

\begin{document}

\section*{Inverting a Matrix through Calculating to RREF}

Given the matrix $A$:
\[
A = \begin{bmatrix}
-4 & 2 & 1 \\
1 & -2 & 3 \\
5 & 1 & 2
\end{bmatrix}
\]

We want to find $A^{-1}$.

\subsection*{Step 1}
Start by augmenting $A$ with the identity matrix:

\[
\left[\begin{array}{ccc|ccc}
-4 & 2 & 1 & 1 & 0 & 0 \\
1 & -2 & 3 & 0 & 1 & 0 \\
5 & 1 & 2 & 0 & 0 & 1 \\
\end{array}\right]
\]

Now, swap the rows:

\[
\left[\begin{array}{ccc|ccc}
1 & -2 & 3 & 0 & 1 & 0 \\
-4 & 2 & 1 & 1 & 0 & 0 \\
5 & 1 & 2 & 0 & 0 & 1 \\
\end{array}\right]
\]

Perform Gaussian elimination (row reduction) to convert the left side to the identity matrix. After performing RREF operations, we get:

\[
\left[\begin{array}{ccc|ccc}
1 & 0 & 0 & \frac{1}{13} & -\frac{3}{13} & \frac{4}{13} \\
0 & 1 & 0 & 1 & -1 & 1 \\
0 & 0 & 1 & -\frac{9}{13} & \frac{14}{13} & -\frac{10}{13} \\
\end{array}\right]
\]

\subsection*{Conclusion}
The left-hand side is now the identity matrix, and the right-hand side is the inverse of $A$:

\[
A^{-1} = \begin{bmatrix}
\frac{1}{13} & -\frac{3}{13} & \frac{4}{13} \\
1 & -1 & 1 \\
-\frac{9}{13} & \frac{14}{13} & -\frac{10}{13}
\end{bmatrix}
\]

If we cannot reduce to RREF, the matrix is not invertible.

\hfill
\section*{Find $A^{-1}$}

We are solving the system $AX + B = CX$. We want to express it as:

\[
B = CX - AX
\]
\[
B = (C - A)X
\]

Thus,
\[
(C - A)^{-1}B = X
\]
We need to compute $C - A$.

\[
C - A = \begin{bmatrix}
1 & 2 \\
2 & 1
\end{bmatrix}
-
\begin{bmatrix}
2 & 1 \\
0 & 6
\end{bmatrix}
=
\begin{bmatrix}
-1 & 1 \\
2 & -5
\end{bmatrix}
\]

Now, we compute the inverse of $C - A$.

To find $(C - A)^{-1}$, we first compute the determinant:

\[
\text{det}(C - A) = (-1)(-5) - (2)(1) = 5 - 2 = 3
\]

Thus,

\[
(C - A)^{-1} = \frac{1}{3}
\begin{bmatrix}
-5 & -1 \\
-2 & -1
\end{bmatrix}
=
\begin{bmatrix}
-\frac{5}{3} & -\frac{1}{3} \\
-\frac{2}{3} & -\frac{1}{3}
\end{bmatrix}
\]

Next, we compute:

\[
(C - A)^{-1}
\begin{bmatrix}
0 & 3 \\
2 & 1
\end{bmatrix}
=
\begin{bmatrix}
-\frac{5}{3} & -\frac{1}{3} \\
-\frac{2}{3} & -\frac{1}{3}
\end{bmatrix}
\begin{bmatrix}
0 & 3 \\
2 & 1
\end{bmatrix}
\]

Performing the matrix multiplication:

\[
\begin{bmatrix}
-\frac{5}{3}(0) + -\frac{1}{3}(2) & -\frac{5}{3}(3) + -\frac{1}{3}(1) \\
-\frac{2}{3}(0) + -\frac{1}{3}(2) & -\frac{2}{3}(3) + -\frac{1}{3}(1)
\end{bmatrix}
=
\begin{bmatrix}
-\frac{2}{3} & -\frac{16}{3} \\
-\frac{2}{3} & -\frac{7}{3}
\end{bmatrix}
\]

Thus, the solution is:

\[
X = \begin{bmatrix}
-\frac{2}{3} & -\frac{16}{3} \\
-\frac{2}{3} & -\frac{7}{3}
\end{bmatrix}
\]

\section*{Calculating the Inverse of Matrices}

\[
\begin{bmatrix}
1 & 3 & 4 \\
0 & 1 & 5 \\
0 & 0 & 1
\end{bmatrix}
\begin{bmatrix}
8 & 10 \\
1 & 5
\end{bmatrix}
=
\begin{bmatrix}
7 & 8 \\
2 & 1 \\
3 & 1
\end{bmatrix}
\]

Let \( A \), \( B \), and \( C \) represent the matrices on the left, middle, and right, respectively.

\subsection*{Getting Inverse of \( A \) using Minors}
The formula for the inverse of a matrix is:

\[
A^{-1} = \frac{1}{\det(A)} \text{adj}(A)
\]

For this matrix, \( \det(A) = 1 \), so:

\[
A^{-1} = 1 \cdot \text{adj}(A)
\]

We create the submatrices of \( A \):

\[
\begin{bmatrix}
+ & - & + \\
- & + & - \\
+ & - & +
\end{bmatrix}
\]

For each cofactor:

\[
+ \begin{vmatrix} 1 & 5 \\ 0 & 1 \end{vmatrix}
- \begin{vmatrix} 0 & 5 \\ 0 & 1 \end{vmatrix}
+ \begin{vmatrix} 0 & 1 \\ 0 & 0 \end{vmatrix}
\]

\[
- \begin{vmatrix} 3 & 4 \\ 0 & 1 \end{vmatrix}
+ \begin{vmatrix} 1 & 4 \\ 0 & 1 \end{vmatrix}
- \begin{vmatrix} 1 & 3 \\ 0 & 0 \end{vmatrix}
\]

\[
+ \begin{vmatrix} 3 & 4 \\ 1 & 5 \end{vmatrix}
- \begin{vmatrix} 1 & 4 \\ 0 & 5 \end{vmatrix}
+ \begin{vmatrix} 1 & 3 \\ 0 & 1 \end{vmatrix}
\]

\subsection*{Getting the Determinants of Each Submatrix}

The resulting matrix is:

\[
\begin{bmatrix}
1 & 0 & 0 \\
-3 & 1 & 0 \\
11 & -5 & 1
\end{bmatrix}
\]

After transposing, we obtain the inverse of \( A \):

\[
A^{-1} = \begin{bmatrix}
1 & -3 & 11 \\
0 & 1 & -5 \\
0 & 0 & 1
\end{bmatrix}
\]

\subsection*{Solving \( X = A^{-1}CB^{-1} \)}

After a few steps of solving:

\[
\begin{bmatrix}
1 & -3 & 11 \\
0 & 1 & -5 \\
0 & 0 & 1
\end{bmatrix}
\begin{bmatrix}
7 & 8 \\
2 & 1 \\
3 & 1
\end{bmatrix}
\begin{bmatrix}
\frac{1}{6} & -\frac{1}{3} \\
-\frac{1}{30} & \frac{4}{15}
\end{bmatrix}
=
\begin{bmatrix}
\frac{77}{15} & -\frac{106}{15} \\
-\frac{61}{30} & \frac{49}{15} \\
\frac{7}{15} & -\frac{11}{15}
\end{bmatrix}
\]

\end{document}
